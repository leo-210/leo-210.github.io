\documentclass{beamer}

\title{Présentation TIPE}
\subtitle{Trouver l'équation différentielle d'un circuit électronique
algorithmiquement.}
\author{Léo Kosman}
\institute{MP2I \textemdash{} Lycée Thiers}
\date{}

\begin{document}
    \begin{frame}
        \titlepage
    \end{frame}

    \begin{frame}
        \frametitle {Idée générale}

        On considère les circuits électroniques composés de dipôles (résistors,
        compdensateurs, bobines et générateurs idéaux de tension et
        d'intensité). 
        \vspace{10}
        \pause
        On veut créer un algorithme qui peut trouver une équation différentielle
        vérifiée par une grandeur du circuit. Et si l'on peut choisir la
        grandeur arbitrairement, c'est encore mieux !
        \vspace{10}
        
        \pause
        Note : on veut trouver l'équation différentielle littéralement, et pas
        résoudre le circuit numériquement.
    \end{frame}

    \begin{frame}
        \frametitle{Quelques résultats obtenus}

        On peut résoudre certains circuit (assez simples) avec la méthode dite
        de "proche en proche". Cela consiste à faire des substitutions répétées.
        \vspace{4}
    
        \pause
        L'algorithme termine et est correct à condition de toujours trouver une
        substitution à chaque étape.
        
        
        \pause
        Ses avantages : il permet de choisir la grandeur qu'on veut, et est
        relativement efficace ($O(n^2)$ a priori, si bien implémenté)
        \vspace{4}

        \pause
        J'ai une implémentation en OCaml, mais elle n'est pas très performante,
        car je n'ai pas spécialement réfléchi à des structures de données plus
        adaptées dans certains cas.
        \vspace{4}

    \end{frame}
    
    \begin{frame}
        \frametitle{Pistes de recherche et objectifs}

        \begin{itemize}
            \item Implémenter les associations en série/parallèle des circuits
                pour trouver une équation différentielle
            \pause
            \item Faire une variante du proche en proche sans choix de la
                grandeur, mais qui permettrait de garantir une équation
                différentielle ?
            \pause
            \item Essayer de trouver un algorithme utilisant le théorème de 
                Millman
            \pause
            \item Optimiser l'implémentation du proche en proche en choisissant
                des structures de données plus adaptées.
        \end{itemize}
    \end{frame}
\end{document}
